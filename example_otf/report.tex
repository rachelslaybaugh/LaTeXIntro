\documentclass[11pt]{article}

%=============================================================================
% Everything between the '=' is the preamble.
%
% This is where document metadata is declared, as well as where we pull in any
% packages we might need for our document.
%
% Let's start with metadata
\title{Example Lab Report}
\author{Ross B.}
\date{\today}
%=============================================================================

\begin{document}
\maketitle

\section{Introduction}
We're pretending to write a lab report to help us learn the basics of \LaTeX.
Let's use Arthur H. Compton's work on inelastic photon scattering as an 
example.

\subsection{History}
A.H. Compton was a physicist who, in 1927, shared the Nobel prize in physics 
for the ``discovery'' of inelastic photon scattering (a.k.a. Compton
scattering).
\subsubsection*{Childhood}
A.H. Compton was born in Wooster, Ohio in the 1890's... 
\subsubsection*{WWII}
He was also a bigshot in the Manhattan Project

\section{Theory}
\LaTeX has become the standard markup for mathematical typesetting. 
Inline math can be put between dollar signs like so: $E = pc$.
A more compelling example might be the Gaussian distribution:
$f(x) = \frac{1}{\sqrt{2\pi\sigma}} e^{\frac{(x - \mu)^2}{2\pi\sigma^2}}$

Sometimes you want
\section{Methods}
\section{Results}
\section{Conclusions}
\section{References}

\end{document}
